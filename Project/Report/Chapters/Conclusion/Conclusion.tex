\chapter{نتیجه‌گیری} 
با توجه به غیر خطی بودن سیستم وضعف سیستم پلنت از لخاظ پایداری که امکان طراحی کنترل‌کننده PID مناسب را غیرممکن میکرد،نیاز بر ان شد تا کنترل کننده پایدارسازی به سیستم اضافه یه مرحله حلقه برای افزایش سیستم بسته شده و سپس کنترل‌کننده PID برای سیتم جدید که پایداری ان بهبود بخشیده شده است به روش های مختلف خواسته شده طراحی شد که خواسته های مسیله را بدون در نظرگرفتن اشباع به خوبی در همه کنترل‌کننده ها ارضا می کند اما مشکل اصلی در حد اشباع عملگر می باشد که با توجه به تابع تبدیل نه چندان مناسب پلنت ،تابع تبدیل پایدارساز طراحی باعث افزایش بیش از حد سیگنال کنترلی می شود که این خود موجب اشباع عملگر در حقیقت می شود و باعث می شود که با افزودن بلوک اشباع به سیستم پاسخ ان تغییر کرده و مقدار زمان نشست ان بسیار افزایش یابد. بدین سبب امکان دست یابی به مقدار زمان نشست مطلوب در هیچ یک از کنترل‌کننده های طراحی شده در صورت استفاده از بلوک اشباع عملگر نمی باشد.